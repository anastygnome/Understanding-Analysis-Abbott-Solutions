%% 2.3 %%

\section{The Algebraic and Order Limit Theorems}
\setcounter{exercise}{0}

\bx{
Let $\epsilon > 0$. Consider $n \geq 1$, then 
\begin{equation*}
	\abs{a - a} = 0 < \epsilon.
\end{equation*}
}

\bx{
\ea{
\item  We are given $(x_n) \rightarrow 0$, so we can make $\abs{x_n - 0}$ as small as we want.

In particular, for some $\epsilon > 0$, we choose $N$ such that $\forall n \geq N$,
\begin{equation}
	\abs{x_n} < \epsilon^2 \quad \Rightarrow \quad \abs{\sqrt{x_n}} < \epsilon
	\label{eq:chap2_sqrt_ineq}
\end{equation}
The implication follows since we know $x_n \geq 0, \epsilon > 0$.

To see that this $N$ works, observe that for all $n \geq N$, 
\begin{equation*}
    \abs{\sqrt{x_n} - 0} < \epsilon \tag{by (\ref{eq:chap2_sqrt_ineq})}
\end{equation*}
so we conclude \((\sqrt{x_n}) \rightarrow 0\).

\label{chap2:converging_zero}

\item We have two cases. If the sequence converges to 0, then we just have part \ref{chap2:converging_zero}.

If $x \neq 0$, then notice
\begin{equation*}
	\abs{
		\sqrt{x_n} - \sqrt{x}
	} = \frac{
		\abs{x_n - x}
	}{
		\abs{\sqrt{x_n} + \sqrt{x}}
	}
\end{equation*}
since we know $x_n \geq 0$ and $x \neq 0$. 
Now, this expression is hard to bound when the denominator is small, since that 
would make the overall expression big. Fortunately, we can put a bound on the denominator,
namely, since we know $x \neq 0 \rightarrow x > 0$, the denominator is $> 0$. Let us call the 
denominator value $d$. Then the following $N$ will work for the convergence proof,
\begin{equation}
	N : \forall n \geq N \quad \abs{x_n - x} < \epsilon \cdot d
\end{equation}
}
}

\bx{
By the Order Limit Theorem, since 
\begin{align*}
	\forall n, x_n \leq y_n \Rightarrow \lim_{n\to\infty} y_n \geq \lim_{n\to\infty} x_n = l \\
	\forall n, z_n \leq y_n \Rightarrow \lim_{n\to\infty} y_n \leq \lim_{n\to\infty} z_n = l
\end{align*}
so $l \leq \lim_{n\to\infty} y_n \leq l \Rightarrow \lim_{n\to\infty} y_n = 1$.
}

\bx{
\AFSOC $\lim a_n = l_1$ and $l_2$, for $l_1 \neq l_2$. Then we have that $\forall \epsilon > 0$, for sufficiently large $n$, that
\begin{align*}
	\abs{a_n - l_1} < \epsilon \\e
	\abs{a_n - l_2} < \epsilon
\end{align*}
But this is a contradiction, since if we let $d = \abs{l_1 - l_2}$, and $\epsilon = \frac{d}{2}$, then 
\begin{align*}
	\abs{l_2 - l_1} \leq \abs{a_n - l_1} + \abs{-(a_n - l_2)} < 2 \epsilon \tag{Triangle Inequality}\\
	d \leq \abs{a_n - l_1} + \abs{-(a_n - l_2)} < d,
\end{align*}
which leads to $d < d$. Thus, we must conclude that $l_1 = l_2$, and limits are unique.
}

\bx{
($\Rightarrow$) If $(z_n)$ is convergent to some $l$, then $\forall \epsilon >0$, we have that $\exists N \in \mathbb{N}$ such that for $n \geq N$, that 
\begin{equation}
	\abs{z_n - l} < \epsilon \implies \abs{x_n - l} < \epsilon,  \abs{y_n - l} < \epsilon,
\end{equation}
because $z_n$ appears before or at the same time as $x_n$ and $y_n$ in the sequence.

($\Leftarrow$) If $(x_n), (y_n)$ are both convergent to some limit $l$, then we have for any $\epsilon > 0$, 
$\exists N_x: n_x\geq N_x$ and $\exists N_y: n_y \geq N_y$, that 
\begin{equation}
	\begin{aligned}
		\abs{x_{n_x} - l} &< \epsilon \\
		\abs{y_{n_y} - l} &< \epsilon,
	\end{aligned}
	\label{eq:chap23_xy_converge}
\end{equation}

respectively.

Choose $N_z > 2\cdot\max(N_x, N_y)$. Then for $n_z \geq N_z$, $z_{n_z}$ is either equal to 
$x_i$ for $i > N_x$ or $y_j$ for $j > N_y$. Using (\ref{eq:chap23_xy_converge}), we can
see that
\begin{equation*}
	\abs{z_{n_z} - l} < \epsilon
\end{equation*}
so $(z_n)$ is also convergent to $l$.
}

\bx{
\ea{
	\item By triangle inequality, we have $\abs{\abs{b_n} - \abs{b}} \leq \abs{b_n - b} < \epsilon$,
	so the $N$ that proves convergence for $(b_n)$ will also work for $(\abs{b_n})$.
	\item The converse is not true. Consider the sequence $a_n = (-1)^n$.
}
}

\bx{
\ea{
	\item Since $(a_n)$ is bounded, call $M$ the upper bound of $(a_n)$. Then since $\abs{b_n}$ can get arbitrarily small, we choose $n \geq N$ such that $\abs{b_n} < \frac{\epsilon}{M}$. Then we have
\begin{align*}
	\abs{a_nb_n} &\leq \abs{a_n}\abs{b_n} \\
	&< M \frac{\epsilon}{M} \\
	&< \epsilon.
\end{align*}

	We cannot use the Algebraic Limit Theorem because we are not given that $(a_n)$ necessarily converges.
	\item No. For example, take $a_n = (-1)^n$, $b_n = 3$. This is because we can no longer make $\abs{b_n}$ arbitrarily small.
	\item When $a=0$, we have 
	\begin{equation*}
		\abs{a_nb_n - ab} \leq \abs{b_n}\abs{a_n - a}.
	\end{equation*}
	We can bound $\abs{b_n} \leq M$, and then choose $n$ such that $\abs{a_n - a} < \frac{\epsilon}{M}$. Then, 
	\begin{align*}
	\abs{a_nb_n - ab} &< M\frac{\epsilon}{M} \\
	&< \epsilon.
	\end{align*}
}
}

\bx{
\ea{
	\item $x_n = (-1)^n, y_n = (-1)^{n-1}$. Sum is just $\{0, 0, \dots\}$
	\item \textbf{Impossible}, since if $x_n + y_n$ converges and $x_n$ also converges, 
	we can show that $y_n$ must converge, which is a contradiction. 
	\item $b_n = \frac{1}{n}$
	\item \textbf{Impossible}, since if $b_n$ converges to some $b$, for any $\epsilon > 0$, 
	past some $N$, for $n \geq N$, 
	\begin{equation*}
		\abs{b_n - b} < \epsilon.
	\end{equation*}
	Any $a_n$ that is unbounded will grow in magnitude for larger $n$, so $b_n$ cannot help bound $a_n$.
	\item $a_n = 0, b_n = n$
}
}

\bx{ 
Yes, the strict inequalities will provide an upper and lower bound still.
Sort of like a $\sup, \inf$ of the sequence.
}

\bx{
Since $\abs{a_n}$ gets arbitrarily small, for any $\epsilon > 0$ 
we know $\exists N : n \geq N$ such that, 
\begin{equation}
\abs{b_n - b} \leq \abs{a_n} < \epsilon.
\end{equation}
}

\bx{
Let $\lim x_n = x$. Then, for any $\epsilon_x > 0$, $\exists N_x : n \geq N_x$, 
we have $\abs{x_n - x} < \epsilon_x$.

Now, our goal is, given some $\epsilon_y > 0$, to find some $N_y : n \geq N_y$ so we can bound $y_n$.
The intuition is, since we know $(x_n)$ converges, 
after some point, $x_i$ will be close to the limit $x$.
Our goal is to choose some $N_y$ large enough so the $x_i'$ prior to these $x_i$
are ``averaged out'' enough, so they are essentially gone, and that the weight on the $x_i$
that are close to $x$ is very high.
\begin{align*}
	\abs{y_n - x} 
	&= \abs{
		\frac{1}{n}
		\pbra{
			\sum_{i=1}^{N_x}(x_i - x) +
			\sum_{i=N_x+1}^{N_y}(x_i - x)
		}
	}\\
	&\leq \abs{
		\frac{1}{n}
		\pbra{
			\sum_{i=1}^{N_x}M +
			\sum_{i=N_x+1}^{N_y}\epsilon_x
		}
	} \tag{Let $M$ bound the difference from $x_i$ to $x$.}\\
	&\leq \abs{
		\frac{1}{n}
		\pbra{
			N_x M +
			\pa{N_y - N_x} \epsilon_x
		}
	}\\
	&\leq \abs{
		\frac{N_x}{n}M +
		\epsilon_x
	}  < \epsilon_y
\end{align*}
Now, we have quite a few choices for our $N_y$. One such solution, is 
\begin{itemize}
	\item Given some $\epsilon > 0$
	\item First choose $N_x$ such $n \geq N_x \quad \abs{x_n - x} < \epsilon/2$
	\item Then, choose $N_y > \frac{2N_xM}{\epsilon}$. This means for $n \geq N_y$,
	\begin{align*}
		\abs{y_n - x} 
		&\leq \abs{
			\frac{N_x}{n}M +
			\epsilon_x
		}\\
		& < 
		\abs{
			\frac{
				N_xM
			}{\frac{2N_xM}{\epsilon}}
			+ 
			\epsilon/2
		} \leq \epsilon
	\end{align*}
\end{itemize}

Consider when $x_n = (-1)^n$. $(x_n)$ does not converge but $(y_n)$ does.
}

\bx{
\ea{
	\item Intuitively, the limit should go to 1, since we have $\frac{\infty}{\infty}$. 
	\begin{align*}
		&\lim_{n\to\infty} \lim_{m\to\infty} a_{m,n} = 1 \\
		&\lim_{m\to\infty} \lim_{n\to\infty} a_{m,n} = 0 \\
	\end{align*}
	\item A sequence $(a_{m,n})$ converges to $l$ if for every $\epsilon > 0$, $\exists N \in \mathbb{N}$ such that whenever $n\geq N$, we have that 
	\begin{align*}
		\abs{\lim_{n\to\infty} \lim_{m\to\infty} a_{m,n} - l} &< \epsilon \\
		\abs{\lim_{m\to\infty} \lim_{n\to\infty} a_{m,n} - l} &< \epsilon.
	\end{align*}
	i.e. we approach the same limit no matter what permutation of the index variables we iterate through.

	This definition is motivated by multivariable calculus, but unsure if this makes sense in the context of analysis.
}
}